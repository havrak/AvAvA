\chapter*{Závěr}
\addcontentsline{toc}{chapter}{Závěr}

Ačkoliv jsme původně měli vyšší nároky na náš projekt, které se kvůli špatnému vyhodnocení obtížnosti způsobené malými předchozími zkušenostmi členů týmu s full stack vývojem a špatným dokumentacím závislostí práce, zatím nepodařilo splnit, hodnotíme práci jako úspěch. Kritické části práce fungují, uživatel si může jednoduše vytvořit kontejner. V něm deploynout svojí aplikaci, a vyzkoušet si tak, jaké to je být adminem na linuxovém serveru. Byl tedy vytvořen funkční systém pro vytváření a správu kontejnerů, který může být po menších úpravách nasazen na školní server.

V budoucnosti bychom rádi dokončili všechny funkce, které jsme původně chtěli implementovat. V aktuálním systému jsou jimi práce se Snapshoty kontejnerů, import záloh, zobrazení historie stavů kontejnerů a projektů a systém na dokoupení uživatelských limitů. Je také nutné otestovat, při jakých limitech je možné vytvářet dobré podmínky pro studentské aplikace, aby se maximalizovala efektivita rozdělování zdrojů.

Systém bychom ovšem chtěli rozšířit ještě víc tím, že bychom přidali administrátorský účet a super-administrátorský účet na správu samotné aplikace. Nakonec máme ještě v plánu vytvořit real-time systém pro kooperaci uživatelů na jednom projektu. S již vylepšenou prací bychom se rádi příští zúčastnili soutěže SOČ.


