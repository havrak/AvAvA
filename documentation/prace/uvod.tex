\chapter*{Úvod}
\addcontentsline{toc}{chapter}{Úvod}

Cílem práce bylo postavit systém pro kontejnerizaci linuxového serveru, které by eventuálně mohl být nasazen na školním serveru.
Systém by spravedlivě rozdělil zdroje mezi jednotlivé žáky a poskytl jim již volný prostor, kde mohou testovat svoje aplikace.
Aktuálně totiž bylo problematické jen získat přístup na školní server.

O frontend práce se staral Vladimír, backend byl rozdělen mezi Kryštofa a Josefa.
Kde Josef se převážně věnoval propojení mezi backendem a lxd deamonem, zatímco Kryštof zpracoval integraci mySQL databáze.
Na spojení obou částí backendu a vytvoření route se podíleli oba členové týmu.

Vývoj probíhal na GitHubu -- \url{https://github.com/havrak/AvAvA}, kde je aplikace dostupná pod licencí GPLv3.
Projekt jsem testovali na vlastní VPS.


