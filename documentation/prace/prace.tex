\documentclass[a4paper,oneside,12pt]{report}
\setlength\textwidth{155mm}
\setlength\textheight{247mm}
\setlength\topmargin{-10mm}
\setlength\headheight{0mm}
\setlength\oddsidemargin{05mm}
\setlength\evensidemargin{05mm}
\let\openright=\clearpage


%% Vytváříme PDF/A-2u
\usepackage[a-2u]{pdfx}

%% Přepneme na českou sazbu a fonty Latin Modern
\usepackage[czech]{babel}
\usepackage{lmodern}
\usepackage[T1]{fontenc}
\usepackage{textcomp}

%% Použité kódování znaků: obvykle latin2, cp1250 nebo utf8:
\usepackage[utf8]{inputenc}

%%% Další užitečné balíčky (jsou součástí běžnýsrov.stribucí LaTeXu)
\usepackage{amsmath}        % rozšíření pro sazbu matematiky
\usepackage{amsfonts}       % matematické fonty
\usepackage{amsthm}         % sazba vět, definic apod.
\usepackage{bbding}         % balíček s nejrůznějšími symboly
			   										% (čtverečky, hvězdičky, tužtičky, nůžtičky, ...)
\usepackage{bm}             % tučné symboly (příkaz \bm)
\usepackage{graphicx}       % vkládání obrázků
\usepackage{fancyhdr}				% možnost slylizovat záhlaví
\usepackage{fancyvrb}       % vylepšené prostředí pro strojové písmo
\usepackage{indentfirst}    % zavede odsazení 1. odstavce kapitoly
\usepackage[nottoc]{tocbibind} % zajistí přidání seznamu literatury,
\usepackage{icomma}         % inteligetní čárka v matematickém módu
\usepackage{dcolumn}        % lepší zarovnání sloupců v tabulkách
\usepackage{booktabs}       % lepší vodorovné linky v tabulkách
\usepackage{paralist}       % lepší enumerate a itemize
\usepackage{caption}				%	popisky
\usepackage{dirtree}				% strom souborů
\usepackage{listings}				% vkládání kódu
\usepackage[bottom]{footmisc}  % poznámky pod čarou vespod
\usepackage{bibentry}
\usepackage{xurl}						% umožní url zalomit všude
\nobibliography*

\usepackage{color}
\usepackage{forest}					% na rodokmeny
\usepackage{natbib}
%\setbibentrystyle{author}

\definecolor{pblue}{rgb}{0.13,0.13,1}
\definecolor{pgreen}{rgb}{0,0.5,0}
\definecolor{pred}{rgb}{0.9,0,0}
\definecolor{pgrey}{rgb}{0.46,0.45,0.48}
\renewcommand{\baselinestretch}{1.5}
%%% Údaje o práci

\def\NazevSkoly{Gymnázium, Praha 6, Arabská 14}
% Název oboru včetně počátečního 'Obor'.
\def\NazevOboru{Programování}

% Název práce v jazyce práce (přesně podle zadání)
\def\NazevPrace{Kontejnerizační systém pro školní server}

% Název práce v angličtině
\def\NazevPraceEN{Containerization system for school server}

% Název práce v němčině
\def\NazevPraceDE{Containerisierungssystem für Schulserver }

% Jméno autra
\def\AutorPrace{Havránek Kryštof, Vávra Vladimír, Litoš Josef 3.E }

% Rok odevzdání
\def\RokOdevzdani{2021}
% Měsíc odevzdání
\def\MesicOdevzdani{Duben}

% Vedoucí práce: Jméno a příjmení s~tituly
\def\Vedouci{Ing. Daniel Kahoun}

% Nepovinné poděkování (vedoucímu práce, konzultantovi, tomu, kdo
% zapůjčil software, literaturu apod.)
\def\Podekovani{%
\textbf{Poděkování}
}

% Abstrakt (doporučený rozsah cca 80-200 slov; nejedná se o zadání práce)
\def\Abstrakt{%
Cílem práce je vytvořit kontejnerizační systém pro operační systém GNU/Linux.
Ten by měl v jednoduchém a přehledném uživatelském prostředí umožnit žákům vytvořit si vlastný linuxový server.
Na které si poté mohou hostovat svoje aplikace, webové stránky, či videohry.
Systém by se měl postarat o to, aby zdroje serveru byly spravedlivě rozdělené mezi jednotlivé uživatele.
}

\def\AbstraktEN{%
Goal of the work is to create containerization system for GNU/Linux operating system.
With its help students should be able to create their own private server in a easy to use webUI.
On such server they will be able to host their applications, websites or video games.
System should aslo garantee fair redistribution of server resources.
}
\def\AbstraktDE{%
Ich spreche Deutsch nicht sehr gut.
}
% 3 až 5 klíčových slov (doporučeno), každé uzavřeno ve složených závorkách
\def\KlicovaSlova{%
	{linux}, {lxd}, {lxc}, {kontejnery}, {nodejs}, {REST API}, {react js}
}
\def\KlicovaSlovaEN{%
	{linux}, {lxd}, {lxc}, {containers}, {nodejs}, {REST API}, {react js}
}

\def\KlicovaSlovaDE{%
	{linux}, {lxd}, {lxc}, {Schulprojekt}, {nodejs}, {REST API}, {react js}
}

%% Balíček hyperref, kterým jdou vyrábět klikací odkazy v PDF,
%% ale hlavně ho používáme k uložení metadat do PDF (včetně obsahu).
%% Většinu nastavítek přednastaví balíček pdfx.
\hypersetup{unicode}
\hypersetup{breaklinks=true}

%% Definice různých užitečných maker (viz popis uvnitř souboru)
\include{makra}

%% Titulní strana a různé povinné informační strany
\fancypagestyle{plain}{
\fancyhf{}
\renewcommand{\headrulewidth}{0.4pt}
\renewcommand{\footrulewidth}{0.4pt}
\fancyhead[C]{}
\fancyhead[L]{Ročníková práce -- \NazevSkoly}
\fancyhead[R]{AvAvA}
\fancyfoot[L]{Vypracovali: \AutorPrace	(\NazevOboru)}
\fancyfoot[C]{}
\fancyfoot[R]{\thepage}
}



\begin{document}

%%% Titulní strana práce

\pagestyle{empty}
\hypersetup{pageanchor=false}

\begin{center}

{\LARGE\bfseries\NazevSkoly}

\vspace{-18mm}
\vfill

{\LARGE\NazevOboru}

\vfill

\centerline{\mbox{\includegraphics[height=4cm]{../img/logo.png}}}

\vspace{-8mm}
\vfill

{\bf\Large ROČNÍKOÝ PROJEKT}

\vfill


\vspace{15mm}

{\LARGE\bfseries\NazevPrace}


\vfill


Vypracovali: \hfill \AutorPrace

Vedoucí práce: \hfill \Vedouci

\vspace{15mm}
\MesicOdevzdani \ \RokOdevzdani

\end{center}



\newpage
\hypersetup{pageanchor=true}
\pagestyle{plain}
\pagenumbering{roman}


\openright


\vspace*{\fill}


\noindent
Prohlašuji, že jsme jedinými autory tohoto projektu, všechny citace jsou
řádně označené a všechna použitá literatura a další zdroje jsou v práci uvedené.
Tímto dle zákona 121/2000 Sb. (tzv. Autorský zákon) ve znění pozdějších předpisů uděluji
bezúplatně škole Gymnázium, Praha 6, Arabská 14 oprávnění k výkonu práva na rozmnožování díla
(§ 13) a práva na sdělování díla veřejnosti (§ 18) na dobu časově neomezenou a bez omezení
územního rozsahu.


\vspace{1cm}

\noindent
V ........ dne ............
\hspace{4cm}
Podpisy autorů


\newpage
%
% %%% Poděkování
%
%\openright
%
\noindent
\Podekovani

\newpage


\openright

\vbox to 0.20\vsize{
\setlength\parindent{0mm}
\setlength\parskip{5mm}

Název práce:
\NazevPrace

Autoři:
\AutorPrace

% Vedoucí práce:
% \Vedouci, \KatedraVedouciho

Abstrakt:
\Abstrakt

Klíčová slova:
\KlicovaSlova

% Opakování v angličtině.

Title:
\NazevPraceEN

Authors:
\AutorPrace

Abstract:
\AbstraktEN

Key words:
\KlicovaSlovaEN


% Opakování v němčině.

Titlel:
\NazevPraceDE

Autoren:
\AutorPrace

Abstrakt:
\AbstraktDE

Schlüsselwörter:
\KlicovaSlovaDE

\vss}

\newpage

\openright



\tableofcontents


\newpage

\chapter*{Úvod}
\addcontentsline{toc}{chapter}{Úvod}

Cílem práce bylo postavit systém pro kontejnerizaci linuxového serveru, které by eventuálně mohl být nasazen na školním serveru.
Systém by spravedlivě rozdělil zdroje mezi jednotlivé žáky a poskytl jim volný prostor, kde mohou testovat svoje aplikace.
Aktuálně totiž bylo problematické jen získat přístup na školní server.

Co se týče rozdělení práce, tak frontend práce se staral Vladimír a backend byl rozdělen mezi Kryštofa a Josefa, kde Josef se převážně věnoval propojení mezi backendem a LXD deamonem, zatímco Kryštof zpracoval integraci MySQL databáze.
Na spojení obou částí backendu a vytvoření route se podíleli oba členové týmu.
Všichni tři členové se během práce podíleli na vytváření Open API 3 specifikace určující nejen způsob komunikace mezi frontendem a backendem, ale také i funkce které má projekt vlastně mít.

Jednotlivé části byly vyvíjeny odděleně.
Frontend byl testován na testovacím serveru, který byl vygenerován Swagger editorem, v němž se vytvářela API specifikace.
Backend byl testován za pomocí jednoduchých skriptů umožňující posílat různé dotazy a požadavky.
Komunikace probíhala přes vlastní Discord server a kód byl sdílen pomocí GitHubu -- \url{https://github.com/havrak/AvAvA}, kde je aplikace dostupná pod licencí GPLv3. Pro vývoj kódu byl zřízen VPS server, na kterém jsou umístěny programy potřebné pro fungování backendu.


\pagenumbering{arabic}
\setcounter{page}{1}

\chapter{Základní informace}

Tato kapitola spěšně projde hardware, na kterém je program vybudován, knihovny, které byly použity.
Dále projde technologie, či protokoly, a na závěr čtenáře seznámí se strukturou souborů.

\section{Hardware}

Projekt je postaven na modulární platformě M5Stack, který je postaven na základě platformy ESP32 (velice podobné Arduinu, ale má zabudovaný wifi modul).
Tyto jednotky obsahuje projekt 5 -- voda, GPS, vytápění podlahy/stav baterie, kola a na záver samotná centrála.
Další hardwarové prvky jsou rozepsány níže ve seznamu, a jsou přiřazeny k jednotlivým jednotkám.
Nákresy s obvody nejsou součástí práce, neb připojení je zřetelné z kódu (jsou zapsané GPIO piny).
\begin{itemize}
	\item Centrální jednotka
		\begin{itemize}
			\item NEXTION display intelligent series model NX8048P070-011C, kód kompatibilní i s Enchanted sérií (stačí vypnout videa pro počasí)
			\item LAN M5Stack modul
			\item Switch od firmy Mikrotik
	\end{itemize}
	\item Jednotka pro vodu
		\begin{itemize}
			\item 2 * tepelný senzor DS18B20 -- jeden umístění zvenčí, druhý ponořený do vody. (Jsou zapojeny na busu paralelně)
			\item průtokoměr -- umístění u odtoku z nádrže s vodou
			\item 2 * relé -- jedno ovládá tepelný okruh v nádrži, druhé ventil pro napouštění nádrže
			\item analogový podvodní senzor tlaku -- umístěn v přípojce na vodu, zjišťuje zda je umístěna přípojka
			\item 2 * vodotěsný spínač -- umístěn dole a nahoře v nádrži, určuje zda voda klesla pod kritickou úroveň a zda je nádrž plná
		\end{itemize}
	\item Jednotka pro GPS -- M5Stack GPS module
	\item Jednotka pro vytápění podlahy, stav baterie (Není realizována)
		\begin{itemize}
			\item 8 * tepelný senzor DS18B20 -- v každém okruhu dva senzory
			\item 4 solid state relé -- ovládání okruhů
			\item eventuálně Bezdrátový stejnosměrný wattmetr VAC1030A, není ani zmíněn v kódu
			\item hall sensory
		\end{itemize}
\end{itemize}

\section{Použité knihovny}

\begin{itemize}
	\item M5Stack -- ovládání displaye na M5Stack -- Dostupné z URL: \url{https://github.com/m5stack/M5Stack}
	\item WiFi -- WiFi, nutné pro protokol ESP-NOW -- Dostupné z URL: \url{https://dl.espressif.com/dl/package_esp32_index.json}
	\item esp\_now -- ESP-NOW -- Dostupné z URL: \url{https://dl.espressif.com/dl/package_esp32_index.json}
	\item EEPROM -- zprovoznění EEPROM -- Dostupné z URL: \url{https://dl.espressif.com/dl/package_esp32_index.json}
	\item TinyGPS++ -- zpracování dat z GPS -- Dostupné z URL: \url{http://arduiniana.org/libraries/tinygpsplus/}
	\item Time -- trackování času -- součást standardní knihovny
	\item Timezone -- práci s časovými pásmy, použita jen pro středoevropské -- Dostupné z URL: \url{https://github.com/JChristensen/Timezone}
	\item NTPClient -- zajištění protokolu NTP -- Dostupné z URL: \url{https://github.com/taranais/NTPClient/})
	\item Ethernet2 -- LAN modul k M5Stack -- Dostupné z URL: \url{https://github.com/m5stack/M5Stack/blob/master/examples/Modules/W5500/Ethernet2.zip}
	\item Ethernet2Udp -- zajištění UDP protokolu -- součást Ethernet2
	\item WebServer -- web server -- součást Ethernet2
	\item Nextion -- práci s displayem Nextion -- Dostupné z URL: \url{https://github.com/itead/ITEADLIB_Arduino_Nextion}
	\item espsoftwareserial -- potřebný k knihovně Nextion -- Dostupné z URL: \url{https://github.com/plerup/espsoftwareserial}
	\item OneWire -- knihovna potřebná k DallasTemperature
	\item DallasTemerature -- tepelné čidla -- Dostupné z URL: \url{https://github.com/milesburton/Arduino-Temperature-Control-Library}
	\item PCA9554 a ClosedCube\_TCA9548A -- pro IO extender na jednotce vytápění
	\item ArduinoJson -- parsování jsonu -- Dostupné z URL: \url{https://github.com/bblanchon/ArduinoJson}
\end{itemize}

\section{Použité programy}

Celý kód byl napsán v textovém editoru Vim a v Arduino IDE.
Gui interface na Nextion display byl vytvořen v Nextion editoru.

Vývoj nejvíce komplikovalo právě Arduino IDE.
Prostředí nemá automatické doplnění, ani žádné možnosti rychle přecházet mezi metodami.
Samozřejmě nejsou k dispozici ani debugovací nástroje, takže přehledné výpisy na sériovou liknu jsou nutností.
Tyto debug zprávy jsou v kódu ponechané a většina z nich má následující formát: ZKRATKA NÁZVU TŘÍDY | název metody | samotná správa.

\section{Požité technologie}

Bezdrátovou komunikaci mezi jednotlivými jednotkami zprostředkovává proprietární protokol ESP-NOW, který je vlastní čipu ESP32.
K získání aktuálního počasí centrála používá standardní http protokol, jedná se však o pouhý jeden GET request.
Komunikace se switchem je zprostředkována za pomocí telnetu.
Čas je aktualizován prostřednictvím NTP, v případě potřeby je však možnost použít data z GPS.
Ke zbylé komunikaci slouží sériová linka a GPIO piny.

Většina práce je napsaná v programovacím jazyce C++, přesněji jeho verzi pro Ardunio.
Některý kód je napsán v programovacím jazyce pro Nextion, ten je uložen v .HMI souboru.
Jedná však o několik málo primitivních funkcí, jako třeba přičtení dne.


\section{Struktura souborů práce}

\dirtree{%
.1 Karavan.
.2 CentralUnit -- složka obsahující kód centrální jednotky.
.3 CentralUnit.ino -- hlavní třída.
.3 Connection.h -- zajišťuje ovládání switche MikroTik pomocí \linebreak[4]telentu.
.3 PowerAndHeating.h -- reprezentuje jednotku zajišťující vytápění\linebreak[4] podlahy a kontrolu stavu baterie.
.3 Security.h -- reprezentuje jednotku zajišťující GPS.
.3 UnitAbstract.h -- abstraktní třída, slouží jako šablona pro\linebreak[4] všechny třídy, které reprezentují fyzickou jednotku.
.3 Water.h -- reprezentuje jednotku u vody.
.3 Weather -- spravuje informace o počasí.
.3 Wheels -- reprezentuje možnou jednotku u kol, nebyla vůbec\linebreak[4] provedena instalace do čidel do kol karavanu.
.2 ESPNowSlaveTemplate -- složka obsahující šablonu pro jednotku.
.3 ESPNowSlaveTemplate.ino -- šablona s implementací ESP-NOW.
.2 HeatingUnit -- složka obsahující kód jednotky pro vytápění.
.3 HeatingUnit.ino  -- kód pro jednotku na vytápění.
.2 WaterUnit -- složka obsahující kód jednotky pro vodu.
.3 WaterUnit.ino -- kód pro jednotku pro vodu.
.2 WheelsUnit -- složka obsahující kód jednotky pro kola.
.3 WheelsUnit.ino -- kód pro jednotku pro kola.
.2 NextionFiles -- soubory pro display Nextion.
.3 Backgrounds -- složka s pozadími.
.3 Fonts -- složka s fonty.
.3 Icons -- složka s ikonkami pro Nextion.
.4 UI -- ikonky pro UI, vlastní tvorba.
.4 Weather icons -- ikonky pro počasí, jak obrázky, tak videa\footnotemark.
.3 Caravan.HMI -- soubor s rozložením displaye.
.3 Caraven.tft -- vyexportované rozložení displaye.
}
\footnotetext{Obrázky pochází z \url{https://dribbble.com/shots/1669480-Weather-Icons/attachments/263247}, ikona tornáda z \url{https://www.flaticon.com/free-icons/tornado}, obrázky, které na těchto stránkách, nenajdete vznikli editací. Zakoupeny z \url{https://videohive.net/item/animated-weather-icons/19199689}}

\chapter{Vybrané části programu}

Vzhledem k délce programu a jeho různorodosti, není možné s limitem délky této dokumentace projít všechny funkce programu.
Proto se tato dokumentace zaměří jen na několik vybraných částí.

\section{ESP-NOW}

Jak už bylo zmíněno, bezdrátový přenos dat mezi jednotkami je zprostředkována pomocí protokolu ESP-NOW.
Tento protokol je relativně jednoduché používat, stačí přidat jednotku mezi peery a poté lze jednoduše posílat na danou mac adresu.
Data jsou posílána ve formě struktury, která je stejná jak na centrále, tak na jednotce.
Po přijmutí dat se jednoduše pomocí memcpy překopírují do instance struktury.

Samozřejmě je nutné při spuštění vytvořit síť s centrálou ve středu a k ní připojenými jednotlivými jednotkami.
Centrála musí vědět od jaké jednotky data přichází a musí být schopna poslat data, či konfiguraci (Zatím není použito, bude nutné u podlahy.).

Původně systém fungoval následovně.
Všechny jednotky měli viditelné SSID centrála je postupně všechny kontaktovala.
Jakmile nalezla jednotku bylo jí zasláno číslo 92 (Není v tom žádný význam) a jednotka odpověděla svým kódem -- 101 pro GPS, 102 pro vodu.
Tento systém měl však několik nevýhod.
Zaprvé nahrál li se nový kód do centrály, tak se museli zresetovat všechny jednotky, protože centrála se v nich se smaže až po 5 minutách nečinnosti.
A zadruhé SSID jednotek byly pořád viditelné a systém neumožňoval je schovat.

Proto byl nakonec tento systém obrácen.
Místo toho aby centrála vyhledávala jednotky nyní jednotky vyhledávají centrálu.
Té poté pošlou svůj kód a centrála opět odpoví číslem 92.
Zároveň si jednotky uloží mac centrály do své EEPROM (stejně tak činní i centrála, kdy ukládá mac jednotek do své EEPROM).
Při dalším spuštění nemusí tak mít centrála viditelné SSID a síť se stejně postaví.
Viditelnost SSID se ovládá pomocí tlačítka na M5Stacku, s tím že na rozdíl od předchozí implementace pouze centrála má své SSID viditelné.

Centrála pravidelně posílá všem jednotkám ping, aby si jednotky centrálu nevzali jako neaktivní a nesmazali jí.
Stejně tak centrála si pamatuje poslední čas, kdy přišla zpráva od jednotky a v případě nečinnosti ji smaže.

Jediný problém při vývoji, který stojí za zmínku, je chybová hláška, již vyvolávala knihovní funkce esp\_now\_add\_peer().
Ta z neznámých důvodů vyhazovala hlášku invalid peer, ovšem někdy se stejným vstupním parametrem funkce proběhla v pořádku.
Tento problém vyřešilo až přesunutí instance struktury toAdd na globální proměnou.
Naštěstí tento problém byl vyřešen již na fóru\footnote{\bibentry{CoolForum}}.

\section{Telnet}

Jednou z funkcí projektu je možnost přepínat zdroj internetu -- mezi WiFi access pointem a LTE modemem.
Tuto funkci zajišťuje switch, ke kterému je centrála připojená přes ethernetový kabel.

Původně bylo v plánu napsat celou implementaci protokolu telnet.
Bohužel logika z neznámých důvodů nefungovala.
Pomocí programu wire shark byly proto odposlechnuty jaké příkazy si mezi sebou client a server posílají během párovaní.
Ty poté byly napevno dané do programu.
Program tak jen čeká na určitou posloupnost a jí odpoví určitou posloupností.

Samotné posílání příkazů již nečiní žádné problémy a celý systém funguje.

\section{Čas}

Uživatel má celkem 3 možnosti jak nastavit čas.

Standardně je čas řešen pomocí protokolu NTP pomocí knihovny NTPClient.
Tento způsob samozřejmě vyžaduje internetové připojení, ale to je téměř vždy dostupné.
Navíc po jednom dotazu si čas centrála udržuje.

Případě nedostupnosti sítě lze použít GPS, samozřejmě pouze za předpokladu, že je připojená GPS jednotka.
Čas z GPS není tak tak přesný jako u NTP, protože po přečtení se musí poslat přes EPS-NOW (Třeba na několikátý pokus.), na centrále se musí zpracovat callback (ESP32 má jen jedno vlákno, takže než se zpracuje uběhne nějaký čas.).
Každopádně se jedná o možnou variantu.

V poslední řadě lze čas nastavit manuálně.
Uživatel na stránce s nastavením času zmáčkne tlačítko nastavit, dále nastaví čas a poté zmáčkne tlačítko znovu.
V callbacku, který tato akce na centrále vyvolá se zkontroluje stav tlačítek NTP a GPS a jsou li obě vypnuté čas se nastaví ze zadaného.
V jiném případě bude uživatelem zadaný čas ignorován.
Na Nextionu jsou samozřejmě přidány různé omezení, aby uživatel nemohl zadat třeba 13. měsíc.
Lze však i tak zadat neplatné datum a to tím, že člověk změní měsíc po nastavení dne.
Bohužel programovací jazyk Nextionu je velmi primitivní a nepodporuje ani základní věci jako je tvorba proměnných (Všechny proměnné jsou součástí objektu na display).
Už jen kód na přidání dne má 57 řádků a není možné třeba při změně měsíce udělat aby vyskočila uživateli chybová hláška.
Nebo minimálně by nebyl takový kód zrovna elegantní.

Střídání času je funguje pouze na středoevropské pásmo, jinak si uživatel musí měnit střídání času a časové zóny sám.
Logika všech pravidel na úpravu času je relativně složitá a jsou na ní potřeba stovky souborů.
Ty sice lze mít uložené na SD kartě (Kterou M5Stack podporuje.), ale nejedná se o příliš efektivní řešení, protože parsování souborů je relativně složitý proces na něčem nevýkoném jako je ESP32.
Uživatel si proto na druhé stránce Nextionu může zvolit manuální nastavení posunu času, to je dě pomocí hodinových inkrementů.

\section{Nextion display}

Ovládacím prvkem celého systému je dispaly od Nextionu.
Na něm jsou aktuálně tři stránky -- hlavní, nastavení času a nastevení vytápění.
Stránka na vytápění však aktuálně neplní žádnou funkci.

Posílání dat display je relativně jednoduché.
Pro čtení dat a používání callbacků je nutná knihovna Nextion.
Pomocí ní vytvoříme instance objektů, které odpovídají těm na obrazovce.
U callbacků je ještě nutné přidat instanci do pole nex\_listen\_list a manuálně přiřadit jaká metoda bude zavolána.

Nutno upozornit, že stejně všechny akce na centrále obstarává jen jedno vlákno.
Takže ne vždy uživatelova akce na obrazovce vyvolá okamžitou reakci v programu.
Někdy se může jedna až o sekundy, než bude callback zpracován.
Z tohoto důvodu se malé akce jako přičtení čísla řeší přímo na Nextionu.

Syntaxe jazyku na Nextionu je velmi podobná ostaním mainstreamovým jazykům.
Ale, jak už bylo zmíněno, nedovoluje vytvářet vlastní proměnné.
Z tohoto důvodu jsou například na stránce s nastavení času tři čísla, které mají barvu textu stejnou jako barvu pozadí.
Jedná se o krkolomné řešení, ale jiné neexistuje.
Také jazyk nepodporuje základní věci jako složené podmínky, či možnost provádět operace v podmínkách.

Jediným problémem je, že přes sériovou linku nelze poslat charaktery s diakritikou.
Z tohoto důvodu není v aktuální době sladěný jazyk, kde ovládací prvky na Nextionu jsou v češtině, zatímco data přicházející z centrály v angličtině.

% \begin{figure}[h]
% 	\centering
% 		\includegraphics[height=7.5cm]{../img/mainScreen.png}
% 		\caption[Hlavní obrzovka]{
% 		Hlavní obrazovka
% 	}
% \end{figure}

% \begin{figure}[h]
% 	\centering
% 		\includegraphics[height=7.5cm]{../img/timeSettingScreen.png}
% 		\caption[Obrazovka s nastavením času]{
% 		Obrazovka s nastavením času
% 	}
% \end{figure}
\section{Paměť}

Některé informace si sytém přenáší do svého dalšího spuštění.
K tomu slouží integrovaná EEPROM v ESP-32.
Jedná se o velmi primitivní typ paměti a konstrukce jako soubory nejsou možné.
Všechna data se dokonce musí zapisovat ručně po bytech.
To činní nutným mít paměť rozkouskovanou a vědět jaké bloky patří jaké proměnné.
Zároveň má EEPROM relativně malý počet možných zapsání.
Z tohoto důvodu se při zápisu čte paměť a data se porovnávají, aby nedocházelo k zbytečným zápisům.

\section{Počasí}

Počasí je získáno pomocí API z open weather map.
Odpovědí na http request je json, který se zprasuje pomocí knihovny ArduinoJson.
Zobrazení počasí může být uděláno jak formou videa, tak formou obrázku.
Video je však možné jen pro Enchated Series display.
Mezi styly je možné vybírat si pouze v kódu, a změna vyžaduje flashnutí centrály novým kódem.
Jaké ikona se zobrazí se řídí ID daného počasí, kompletní seznam je dostupný na stránkách open weather map\footnote{\bibentry{CoolCodes}}.

\chapter{Budoucí části programu}

Bohužel nebylo možné kompletně práci dokončit.
Nebyl realizován systém pro kola, z důvodu chybějící instalace, a nutnosti ji prakticky otestovat.
Také nebyla realizována část podlahy a stavu baterie.
Technicky jsou však tyto části již vymyšlené a na GitHubu, kde je práce publikována, budou časem dostupné.
Tato kapitola projde alespoň jejich algoritmickou stránku, i když samotný kód ještě nebyl naspán.

\section{Senzoru u pneumatik}

Je plánované do kol karavanu vložit senzory tlaku a teploty.
Podobný systém je jíž instalovaný v multivanu, které karavan tahá.
Informace ze senzorů se budou stejně jako další zobrazovat na Nextionu, ale zároveň budou posílány i do jednotky v autě, aby byly dostupné pro řidiče i za jízdy.
Bude se jednat o senzory vybavené technologií bluetooth, kterou M5Stick podporuje.

Prakticky je implementace relativně jednoduchá, s tím, že valnou většinu práce řeší knihovna.
Jedinou problémovou stránku je identifikace jednotlivých kol.
Zatím je kód napsaný s napevno danými ID senzorů a kola musí být dána vždy ve stejném pořadí.
Na vysvětlenou mám zjištěný ID předního levého kola a to kolo musí být vždy levé přední, nelze ho dát jinam a pokud by bylo potřeba kolo vyměnit za nové, tak by se musel předělávat kód.
Toto je nadmíru nepraktické.
Teoreticky by nová implementace měla pracovat se sílou signálu.
M5Stick by se nenacházel ve středu auta, ale byl by výrazně blíž jedné pneumatice (Jeho pozice by byla napevno daná.).
Identifikace čidel by poté probíhala pomocí síly signálu.
Bohužel implementace vyžaduje zkušení a ruční dolaďování, neboť vzdálenost čidel se mění (Jsou v kolech, která se otáčí.), navíc některé části auta můžou bránit signálu.
To je potřeba řešit přímo u karavanu, z tohoto důvodu nebyla udělána implementace.

\section{Vytápění podlahy a stav baterie}

Implementace podlahy a stavu baterie nebyla zprovozněna ze stejných důvodů jako pneumatiky -- psaní naslepo bez možnosti některé části testovat je velmi komplikované a prakticky zbytečné.

\subsection{Elektřina}

Implementace elektřiny je relativně jednoduchá.
Původně měla probíhat komunikace s wattmetrem pomocí bezdrátové technologie, bohužel protokol není otevřený veřejnosti a data jsou zašifrovaná.
Komunikace aktuálně funguje pomocí sériové linky přes USB.
Bohužel se také nepodařilo sehnat chtěné hall senzory, z tohoto důvodu je dokončení této části pozastaveno na neurčito.
Nutno podotknou, že jak wattmetrem, tak hall senzory se zabýval hlavně strýc.
Sám jsem se v této části projektu příliš neangažoval.
Z tohoto důvodu nejsou v kódu jednotky vytápění žádné reference na elektřinu a je zmíněna pouze na Nextionu a centrále (Nemá smysl schovávat ).

\subsection{Vytápění podlahy}

U vytápění podlahy se řeší podobný problém jako u kol -- je potřeba identifikovat tepelné senzory, které jsou při každém zapnutí indexované v jiném pořadí.
Ze senzorů lze získat 8 bytové unikátní číslo, které bylo senzoru přiřazeno při výrobě.
Identifikace bude probíhat podobně jako probíhá u vody, kde byl takový systém v mnohem menším měřítku již implementován a odzkoušen.
Vytápěcích okruhů bude 4 až 5 a na každý připadnou právě dvě čidla DS.

Při identifikaci, kterou člověk spustí z Nextionu, se postupně budou zapínat okruhy a bude se měřit $\Delta t$, vždy které senzory přesáhnou 3 (asi 3) stupně, tak se přidají k danému okruhu.
Na rozdíl od vody, je však potřeba více testovaní, protože podlaha se bude zahřívat i tam, kde se zrovna netopí a mohly by tak vzniknout třeba 3 senzory pro jeden okruh atd.
Také je nutné řešit, že některé čidla přestanou fungovat a jejich rychlá výměna je nemožná.
Program vyžaduje nalezení celé řady konstant, navíc velmi záleží na rozmístění čidel, a proto jsem implementaci řešil pouze algoritmicky.

Na vysvětlenou čidla teploty podlahy jsou nutné, aby se podlaha nezahřála nad limitní teplotu.
Nehledě na to, že je praktické mít možnost nastavit si teplotu podlahy a ne pouze teplotu vzduchu.

% \chapter{Obrázky systému}
% \begin{figure}[h]
% 	\centering
% 		\includegraphics[height=8cm]{../img/dataatad.jpg}
% 		\caption[Obrazovka při běhu programu]{
% 		Obrazovka při běhu programu
% 	}
% \end{figure}

% \begin{figure}[h]
% 	\centering
% 		\includegraphics[height=8cm]{../img/sys.jpg}
% 		\caption[Obrazovka s 2 moduly M5Stack]{
% 		Obrazovka s 2 moduly M5Stack
% 	}
% \end{figure}

\chapter*{}
\pagenumbering{roman}
\setcounter{page}{6}
\chapter*{Závěr}
\addcontentsline{toc}{chapter}{Závěr}

Ačkoliv jsme původně měli vyšší nároky na náš projekt, které se kvůli špatnému vyhodnocení obtížnosti způsobené malými předchozími zkušenostmi členů týmu s full stack vývojem a špatným dokumentacím závislostí práce, zatím nepodařilo splnit, hodnotíme práci jako úspěch. Kritické části práce fungují, uživatel si může jednoduše vytvořit kontejner. V něm deploynout svojí aplikaci, a vyzkoušet si tak, jaké to je být adminem na linuxovém serveru. Byl tedy vytvořen funkční systém pro vytváření a správu kontejnerů, který může být po menších úpravách nasazen na školní server.

V budoucnosti bychom rádi dokončili všechny funkce, které jsme původně chtěli implementovat. V aktuálním systému jsou jimi práce se Snapshoty kontejnerů, import záloh, zobrazení historie stavů kontejnerů a projektů a systém na dokoupení uživatelských limitů. Je také nutné otestovat, při jakých limitech je možné vytvářet dobré podmínky pro studentské aplikace, aby se maximalizovala efektivita rozdělování zdrojů.

Systém bychom ovšem chtěli rozšířit ještě víc tím, že bychom přidali administrátorský účet a super-administrátorský účet na správu samotné aplikace. Nakonec máme ještě v plánu vytvořit real-time systém pro kooperaci uživatelů na jednom projektu. S již vylepšenou prací bychom se rádi příští zúčastnili soutěže SOČ.




\include{literatura}

\listoffigures
\openright
\end{document}
